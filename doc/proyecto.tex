%%%%%%%%%%%%%%%%%%%%%%%%%%%%%%%%%%%%%%%%%
% Short Sectioned Assignment LaTeX Template Version 1.0 (5/5/12)
% This template has been downloaded from: http://www.LaTeXTemplates.com
% Original author:  Frits Wenneker (http://www.howtotex.com)
% License: CC BY-NC-SA 3.0 (http://creativecommons.org/licenses/by-nc-sa/3.0/)
%%%%%%%%%%%%%%%%%%%%%%%%%%%%%%%%%%%%%%%%%

% \documentclass[paper=a4, fontsize=11pt]{scrartcl} % A4 paper and 11pt font size
\documentclass[12pt, a4paper]{book}
\usepackage[T1]{fontenc} % Use 8-bit encoding that has 256 glyphs
\usepackage[utf8]{inputenc}
\usepackage{fourier} % Use the Adobe Utopia font for the document - comment this line to return to the LaTeX default
\usepackage{listings} % para insertar código con formato similar al editor
\usepackage[spanish, es-tabla]{babel} % Selecciona el español para palabras introducidas automáticamente, p.ej. "septiembre" en la fecha y especifica que se use la palabra Tabla en vez de Cuadro
\usepackage{url} % ,href} %para incluir URLs e hipervínculos dentro del texto (aunque hay que instalar href)
\usepackage{graphics,graphicx, float} %para incluir imágenes y colocarlas
\usepackage[gen]{eurosym} %para incluir el símbolo del euro
\usepackage{cite} %para incluir citas del archivo <nombre>.bib
\usepackage{enumerate}
\usepackage{hyperref}
\usepackage{graphicx}
\usepackage{tabularx}
\usepackage{booktabs}
\usepackage{glossaries}
\usepackage{placeins}

\usepackage{listings}

\usepackage[table,xcdraw]{xcolor}
\hypersetup{
	colorlinks=true,	% false: boxed links; true: colored links
	linkcolor=black,	% color of internal links
	allcolors = cyan
}


%New colors defined below
\definecolor{codegreen}{rgb}{0,0.6,0}
\definecolor{codegray}{rgb}{0.5,0.5,0.5}
\definecolor{codepurple}{rgb}{0.58,0,0.82}
\definecolor{backcolour}{rgb}{0.95,0.95,0.92}

%Code listing style named "mystyle"
\lstdefinestyle{mystyle}{
  backgroundcolor=\color{backcolour}, commentstyle=\color{codegreen},
  keywordstyle=\color{magenta},
  numberstyle=\tiny\color{codegray},
  stringstyle=\color{codepurple},
  basicstyle=\ttfamily\footnotesize,
  breakatwhitespace=false,         
  breaklines=true,                 
  captionpos=b,                    
  keepspaces=true,                 
  numbers=left,                    
  numbersep=5pt,                  
  showspaces=false,                
  showstringspaces=false,
  showtabs=false,                  
  tabsize=2
}

%"mystyle" code listing set
\lstset{style=mystyle}

\renewcommand{\familydefault}{\sfdefault}
\usepackage{fancyhdr} % Custom headers and footers
\pagestyle{fancyplain} % Makes all pages in the document conform to the custom headers and footers
\fancyhead[L]{} % Empty left header
\fancyhead[C]{} % Empty center header
\fancyhead[R]{Pablo Jiménez Jiménez} % My name
\fancyfoot[L]{} % Empty left footer
\fancyfoot[C]{} % Empty center footer
\fancyfoot[R]{\thepage} % Page numbering for right footer
%\renewcommand{\headrulewidth}{0pt} % Remove header underlines
\renewcommand{\footrulewidth}{0pt} % Remove footer underlines
\setlength{\headheight}{13.6pt} % Customize the height of the header

\usepackage{titlesec, blindtext, color}
\definecolor{gray75}{gray}{0.75}
\newcommand{\hsp}{\hspace{20pt}}
\titleformat{\chapter}[hang]{\Huge\bfseries}{\thechapter\hsp\textcolor{gray75}{|}\hsp}{0pt}{\Huge\bfseries}
\setcounter{secnumdepth}{4}
\usepackage[Lenny]{fncychap}
\linespread{1.25}

\usepackage{glossaries}

\newglossaryentry{mortalidad}
{ name=Mortalidad, description={La cantidad de personas que mueren en un lugar y período
	de tiempo concreto.} }
\newglossaryentry{morbilidad}
{ name=Morbilidad, description={La cantidad de personas que enferman en un lugar y período
	de tiempo concreto.} }
\newglossaryentry{momo}
{ name=gráficas MoMo, description={Son un sistema de monitorización de la mortalidad
	diaria que permite observar gráficamente como la evolución del número de defunciones
	que se producen.} }
\newglossaryentry{CIE-10}
{ name=CIE-10, description={Es el acrónimo de la Clasificación internacional de
    enfermedades, 10.ª edición correspondiente.} }
\newglossaryentry{SNS}
{ name=SNS, description={El Sistema Nacional de Salud de España es el ente que engloba a
    las prestaciones y servicios sanitarios de España.} }
\newglossaryentry{SOLID}
{ name=SOLID, description={Hace referencia a 5 principios de diseño de software. Single
Responsibility Principle, Open/Closed Principle, Liskov Substitution Principle, Interface
Segregation Principle y Dependency Inversion Principle } }
	

\makenoidxglossaries


\NewDocumentCommand{\codeword}{v}{%
\texttt{{#1}}%
}

\setcounter{tocdepth}{4}
\setcounter{secnumdepth}{4}
\begin{document}

	% Plantilla portada UGR
	\begin{titlepage}
\newlength{\centeroffset}
\setlength{\centeroffset}{-0.5\oddsidemargin}
\addtolength{\centeroffset}{0.5\evensidemargin}
\thispagestyle{empty}

\noindent\hspace*{\centeroffset}\begin{minipage}{\textwidth}

\centering
\includegraphics[width=0.9\textwidth]{doc/logos/logo_ugr.jpg}\\[1.4cm]

\textsc{ \Large TRABAJO FIN DE GRADO\\[0.2cm]}
\textsc{ GRADO EN INGENIERÍA INFORMÁTICA}\\[1cm]

\vspace{2cm}

{\Huge\bfseries Computación y optimización en la nube \\}
\noindent\rule[-1ex]{\textwidth}{3pt}\\[3.5ex]
{\large\bfseries Implementación de una aplicación de datos abiertos en la nube }
\end{minipage}


\noindent\hspace*{\centeroffset}
\begin{minipage}{\textwidth}
\centering

\textbf{Autor}\\ {Pablo Jiménez Jiménez}\\[2.5ex]
\textbf{Director}\\ {Juan Julián Merelo Guervós}\\[2cm]
\includegraphics[width=0.3\textwidth]{doc/logos/etsiit_logo.png}\\[0.1cm]
\textsc{Escuela Técnica Superior de Ingenierías Informática y de Telecomunicación}\\
\textsc{---}\\
Granada, 8 de julio de 2022
\end{minipage}
\end{titlepage}


	% Plantilla prefacio UGR
	\thispagestyle{empty}

\begin{center}
{\large\bfseries Computación y optimización en la nube \\ Implementación de una aplicación de datos abiertos en la nube }\\
\end{center}
\begin{center}
Pablo Jiménez Jiménez \\
\end{center}

%\vspace{0.7cm}

\vspace{0.5cm}
\noindent{\textbf{Palabras clave}: \textit{software libre, open source, API, OpenAPI, GraphQL, pandas, prophet, datos abiertos, DDD, cloud architecture.}
\vspace{0.7cm}

\noindent{\textbf{Resumen}\\
\vspace{0.7cm}
\\
En este proyecto se proporciona un sistema de información que permita a usuarios con
ciertos conocimientos informáticos consultar la evolución sobre las causas de muerte en
España desde el año 1980. Distintos organismos gubernamentales están recopilando esta
información pero resulta imposible poder consultar los datos de forma eficiente y por
tanto poder realizar estudios de cualquier índole.  
\vskip 0.2in

A lo largo de este documento se detallará como se ha dado solución a este problema
realizando un desarrollo ágil guiado por las historias de los usuarios. Se documentará la
solución ofrecida y la justificación del diseño realizado. Como resultado del proyecto
tendremos una interfaces de comunicación agnóstica (independiente del lenguaje de
programación) con la que los usuarios podrán obtener los datos que sean de su interés,
posibilitando el filtrado de estos en base a sus variables. Además, se implementan algunas
sencillas operaciones de predicción y generación de gráficos.


\clearpage

\begin{center}
	{\large\bfseries Cloud computing and optimization \\ Implementation of an open data application in the cloud }\\
	
\end{center}
\begin{center}
	Pablo Jiménez Jiménez \\
\end{center}
\vspace{0.5cm}
\noindent{\textbf{Key words}: \textit{open source, API, OpenAPI, GraphQL, pandas, prophet, open data, DDD, cloud architecture.}
\vspace{0.7cm}

\noindent{\textbf{Abstract}\\
\vspace{0.7cm}
\\
This project provides an information system that allows users with certain computer skills
to consult the evolution of the causes of death in Spain since 1980. Different government
agencies collect this information but it is currently impossible to consult the data
efficiently and therefore to carry out studies of any kind.
\vskip 0.2in

Throughout this document, it will be detailed how a solution has been found by carrying
out an agile development guided by the described user stories. The solution offered and
the justification for the design are documented. As a result of the project, we will have
an agnostic communication interface (independent of the programming language) with which
users will be able to obtain the data they are interested in, making it possible to filter
them based on their variables. In addition, some simple prediction and graphic generation
operations are implemented.

\cleardoublepage

\thispagestyle{empty}

\noindent\rule[-1ex]{\textwidth}{2pt}\\[4.5ex]

D. \textbf{Juan Julián Merelo Guervós}, Profesor(a) del departamento de Arquitectura y Tecnología de Computadores.

\vspace{0.5cm}

\textbf{Informo:}

\vspace{0.5cm}

Que el presente trabajo, titulado \textit{\textbf{Computación y optimización en la nube:
Implementación de una aplicación de datos abiertos en la nube}}, ha sido realizado bajo mi
supervisión por \textbf{Pablo Jiménez Jiménez}, y autorizo la defensa de dicho trabajo
ante el tribunal que corresponda.

\vspace{0.5cm}

Y para que conste, expiden y firman el presente informe en Granada a 8 de julio de 2022.

\vspace{1cm}

\textbf{El director: }

\vspace{5cm}

\noindent \textbf{Juan Julián Merelo Guervós}

\chapter*{Agradecimientos}

A mi familia por estar ahí. 

A mis padres por inculcarme desde pequeño la importancia de formarme y por haberlo hecho posible.

	% Índice de contenidos
	\newpage
    {\hypersetup{hidelinks}
    \tableofcontents
    }

	% Índice de imágenes y tablas
	\newpage
	{\hypersetup{hidelinks}
    \listoffigures
    }
	

	% Si hay suficientes se incluirá dicho índice
	
	{\hypersetup{hidelinks}
    \listoftables 
    }
	\newpage

	\printnoidxglossaries


	% Introducción 
	\chapter{Introducción}
Este proyecto es software libre, y está publicado con la licencia \cite{gplv3} General
Public License v3. Se puede acceder a través de GitHub en este
\href{https://github.com/pablojjimenez/TFG}{enlace} puedes sentirte libre de contribuir
mediante una solicitud de fusión o \href{https://github.com/pablojjimenez/TFG/pulls}{Pull
Request}. También forma parte de los \href{https://github.com/JJ/TF-libres-UGR}{trabajos
liberados} de la UGR.

\section{Motivación} 
El tratamiento automático de la información por medio de técnicas matemáticas procesadas
por un ordenador ha cambiado la forma en la que nos organizamos, estudiamos y obtenemos
conclusiones.

En particular, los servicios sanitarios suponen un eje vertebrador cuando hablamos de
mejorar la calidad de vida de las personas. Numerosos estudios sociológicos avalan que la
sanidad es una preocupación incesante de los españoles, más aún si cabo tras la pandemia
que hemos atravesado. Como podemos observar en la gráfica \ref{fig:problemascis}, entre
las preocupaciones de los españoles la sanidad siempre se muestra con una tendencia al
alta a pesar de no estar exenta de movimientos sinusoidales. Esta tendencia solo es
superada por el paro y problemas relacionados de índole económico.

\FloatBarrier
\begin{figure}[h]
	\centering	
	\includegraphics[width=\textwidth]{doc/logos/imgs/CIS_1.png}
	\caption{ Principales problemas de los españoles según el CIS.
        \href{https://www.rtve.es/noticias/20201015/crisis-economica-coronavirus-preocupan-ahora-mas-espanoles-paro/2045610.shtml}{Enlace
        al artículo} }
    \label{fig:problemascis}
\end{figure}
\FloatBarrier

Vivimos una época donde el almacenamiento de datos es una actividad transversal incesante
que se puede observar en todas las vertientes. Sin embargo, si está información no está
publicada para que se pueda acceder de forma estructurada a ella no nos sirve de nada. 

Los datos que ahora mismo almacena el Instituto de Salud Carlos III\footnote{institución de
referencia científica para el Sistema Nacional de Salud} sobre las defunciones son inaccesibles.  Mi
fundamentación es facilitar los estudios sobre la evolución de las defunciones según causa
de muerte haciendo accesibles los datos públicos que se revelan anualmente para que
programadores e investigadores puedan crear productos que mejoren la prevención del sistema sanitario.

Ahora mismo es imposible poder consultar los datos mencionados para obtener información
concreta y útil.  Si ni podemos consultar estos datos por variables, muy lejos estamos de
poder realizar visualizaciones o tener aplicaciones que sean capaces de obtener valor a
partir de estos datos.

\section{Objetivos}
\label{sec:obj}
El objetivo principal de este proyecto es que el sistema sea capaz de darle a los
programadores e investigadores interesados la capacidad de conocer la evolución de las causas de muerte. 

La finalidad es deparar de una interfaz uniforme que permita realizar consultas
sobre los datos almacenados utilizando un protocolo agnóstico actual. Además, la
capacidad de generar gráficos con la evolución y someros pronósticos sobre la evolución 
de las defunciones es un complemento que aporta un valor indispensable. Las causas de
muerte se encuentran reservadas según la Clasificación Internacional
de Enfermedades y se encuentran clasificadas en base a distintas variables. 

El sistema debe de atender a las necesidades de los usuarios diana a utilizar el sistema, para
identificarlos y poder trabajar con un desarrollo ágil se ha utilizado la
\hyperref[sec:usu]{herramienta de personas ficticias.}

\section{Usuarios}
\label{sec:usu}
Una de las herramientas que nos permiten definir el alcance y las necesidades de la
aplicación es la creación de perfiles de personas ficticias, candidatos diana a utilizar
el sistema en un futuro.

De esta forma, podemos justificar las decisiones tomadas en aras de alcanzar el objetivo
por parte de los usuarios interesados en el producto final. La función de estos usuarios
es poder crear posteriormente historias de usuario que identifiquen los problemas y deseos
de los usuarios para que posteriormente en equipo de desarrollo pueda proponer una
solución.

He creado dos someras personalidades lo más variopintas posibles en aras de enfocarnos más
aún en el usuario y dotar de mayor calidad el resultado final, de modo que podamos abarcar
por completo las necesidades de estos.

\rowcolors{1}{gray!20}{gray!8}
\begin{table}[H]
   \begin{center}
      \begin{tabular}{| p{\dimexpr 0.25\linewidth-2\tabcolsep} | p{\dimexpr
                   0.8\linewidth-2\tabcolsep} |}
         \hline
         Persona 1 &  \\ \hline
         \textbf{Nombre} & Raquel \\
         \textbf{Edad} & 30 años \\
         \textbf{Formación} & Senior frontend developer \\
         \textbf{Personalidad} & \begin{itemize}
                \item Deportista.
                \item Tímida.
                \item Lógica. \end{itemize} \\
         \textbf{¿Cuál es su entorno?} & \begin{itemize}
                \item Sus amigos.
                \item Su gato.
                \item Sus libros. \end{itemize} \\
         \textbf{¿Qué dispositivos utiliza en su día a día?} & \begin{itemize}
                \item Un portátil.
                \item Un smartphone. \end{itemize} \\
            \textbf{¿Cuál es su actitud hacía la tecnología?} & \begin{itemize}
                \item Perezosa.
                \item Sabe programar. \end{itemize} \\
            \hline
      \end{tabular}
      \caption{Usuario ficticio 2}
   \end{center}
\end{table}

\rowcolors{1}{gray!20}{gray!8}
\begin{table}[H]
   \begin{center}
      \begin{tabular}{| p{\dimexpr 0.25\linewidth-2\tabcolsep} | p{\dimexpr
                   0.8\linewidth-2\tabcolsep} |}
         \hline
         Persona 3 &  \\ \hline
         \textbf{Nombre} & Isabel \\
         \textbf{Edad} & 45 años \\
         \textbf{Formación} & Matemática \\
         \textbf{Personalidad} & \begin{itemize}
                \item Aventurada.
                \item Protagonista. \end{itemize} \\
         \textbf{¿Cuál es su entorno?} & \begin{itemize}
                \item Su familia.
                \item Sus padres.
                \item Su trabajo. Profesora en la UGR. \end{itemize} \\
         \textbf{¿Qué dispositivos utiliza en su día a día?} & \begin{itemize}
                \item Un ordenador de sobremesa.
                \item Un smartphone.
                \item Un iPad que comparte con su marido. \end{itemize} \\
            \textbf{¿Cuál es su actitud hacía la tecnología?} & \begin{itemize}
                \item Valiente.
                \item Proactiva. \end{itemize} \\
            \hline
      \end{tabular}
      \caption{Usuario ficticio 2}
   \end{center}
\end{table}


	% Estado del arte
	%   1. Crítica al estado del arte
	%   2. Propuesta
	\chapter{Estado del arte}

En este apartado lo que pretende es mostrar como se encuentra el panorama en el cual vamos a llevar a cabo este proyecto. Para ello, en primer lugar voy a analizar soluciones similares ya existentes, de forma que analicemos los requisitos de nuestro sistema comparándolo con aquellos que ya existen para obtener nuestra propuesta de valor.

\section{Análisis de mercado}

Como he comentado ya con anterioridad, el acceso a los datos sobre las defunciones es de dominio público, se puede encontrar información al respecto en los sitios que voy a exponer a continuación. Cualquiera puede acceder a ellos y verlos, el principal problema surge cuando queremos hacernos preguntas en base a esos números que vemos. Hasta que nivel se adaptan los sistemas existentes a las necesidades de mis usuarios. Vamos a ver las principales fuentes que existen y lo que nos ofrecen para a continuación, saber que tipo de tratamiento informático necesitamos realizar para satisfacer las necesidades de estos.

\subsection{Instituto Nacional de Estadística}
Uno de los primeros sitios en los que consulté fue la página del Instituto Nacional de Estadística\footnote{https://www.ine.es/index.htm} que es un referente en España en cuanto a la coordinación de servicios estadístico.
En este \hyperref{enlace}{https://www.ine.es/jaxiT3/Tabla.htm?t=6609} \footnote{https://www.ine.es/jaxiT3/Tabla.htm?t=6609} se pueden observar las defunciones según causa de muerte filtradas por causa, sexo, edad y periodo. 
Enumero las principales desventajas encontradas desde el punto de vista de mis usuarios:
\begin{itemize}
    \item No permite realizar consultas a los datos.
    \item No soporta ningún protocolo para obtener agnósticamente estos datos y poder usarlo en otros softwares/aplicaciones.
    \item Poca precisión de filtrado por las pocas variables de las que disponemos.
    \item No tenemos visualización gráfica. Te redirige a descargar el programa PC-Axis únicamente disponible para Windows si quieres ver mas detalles sobre los datos almacenados.
\end{itemize}
\FloatBarrier
\begin{figure}[]
	\centering
	\includegraphics[scale=0.5]{doc/logos/imgs/ine1.png}
	\caption{  Vista principal para obtener datos en el INE }
    \label{fig:worst_f_value}
\end{figure}

\begin{figure}[]
	\centering
	\includegraphics[scale=0.5]{doc/logos/imgs/ine2.png}
	\caption{ Vista resultado cuando se han filtrado los datos a obtener }
    \label{fig:worst_f_value}
\end{figure}
\FloatBarrier

\subsection{Instituto de Salud Carlos III}
En la página del Instituto de Salud Carlos III \footnote{https://www.isciii.es/Paginas/Inicio.aspx} podemos encontrar muchas entradas  hablando sobre \Gls{mortalidad} de la población y \Gls{morbilidad} apoyada de \Gls{momo} \hyperref{Instituto de Salud Carlos III - Vigilancia de la Mortalidad Diaria}{https://www.isciii.es/QueHacemos/Servicios/VigilanciaSaludPublicaRENAVE/EnfermedadesTransmisibles/MoMo/Paginas/default.aspx} que son gráficas de desviaciones de mortalidad diaria respecto a la esperada según las series históricas de mortalidad, estas constituyen un sistema de vigilancia que proporciona información sobre el impacto en mortalidad de la población. Estas son entregables en formato web y PDF.

A pesar de la valiosísima información que estos informes pueden arrojar, sobre todo desde el punto de vista sanitario. Esto no es lo que persigue satisfacer en este trabajo, que se centra en facilitar la obtención de datos crudos. El instituto pone a disposición de los ciudadanos dos servidores.

\subsubsection{Servidor Arïadna}
Nos permite consultar información sobre las causas de defunción atendiendo a cuatro variables:
\begin{enumerate}
  \item Indicador.
  \item Causa.
  \item Período, años.
  \item Sexo.
  \item Comunidad Autónoma.
  \item Provincia.
\end{enumerate}

Y los indicadores pueden ser:
\begin{enumerate}
  \item Tasa ajustada a la población europea.
  \item Tasa ajustada a la población mundial.
  \item Tasa truncada: tasa ajustada de mortalidad limitada a los 35-64 años de edad.
  \item Índice comparativo de mortalidad: Es el cociente entre la tasa ajustada por edad en cada provincia y la tasa
  ajustada para el conjunto de España.
  \item Tasa cruda: la tasa cruda de mortalidad es la proporción de personas que fallecen con respecto al total
  de la población en un período determinado. Se expresa habitualmente como el número de defunciones al año
  por cada 100.000 personas.
  \item Número de defunciones
\end{enumerate}
\FloatBarrier
\begin{figure}[]
	\centering
	\includegraphics[scale=0.5]{doc/logos/imgs/ariadna1.png}
	\caption{ Vista principal del servidor Arïadna \footnote{http://ariadna.cne.isciii.es/} }
    \label{fig:worst_f_value}
\end{figure}

\begin{figure}[]
	\centering
	\includegraphics[scale=0.5]{doc/logos/imgs/ariadna2.png}
	\caption{ Vista por mortalidad provincial del servidor Arïadna }
    \label{fig:worst_f_value}
\end{figure}
\FloatBarrier
Los datos son ofrecidos mediante un mapa de España interactivo podemos observar también gráficas de todas las provincias y la media nacional así como una tabla con los datos. Es de obligado comentar la antigüedad manifiesta de la página y la poca agilidad con la que cargan los contenidos.
Esta página permite descargarse los datos únicamente en formato CSV y sólo de 10 tuplas en 10 tuplas sobre las  variables filtradas. Lo que sigue dificultando el acceso completo a cualquier interesado.

\subsubsection{Servidor Raziël}
Es un servidor interactivo \footnote{http://raziel.cne.isciii.es/index.php} que genera mapas y gráficas de España por comunidades autónomas, y tablas de datos que muestran las diferencias en la mortalidad por diversas causas. Ofrece datos desde el año 1980, de acuerdo con los criterios que de el usuario.
La web nos permite consultar las gráficas citadas anteriormente atendiendo a cuatro variables:
\begin{enumerate}
    \item Indicador.
    \item Causa.
    \item Período, años.
    \item Sexo.
    \item Grupo de edades.
    \item Comunidad Autónoma.
\end{enumerate}

Esta web es prácticamente inaccesible e imposible de obtener los datos en tablas o en algún formato descargable por lo que no me quedó otra opción más que contactar con los servicios informáticos del instituto.


\section{Mi propuesta ante el estado del arte}

Sin perder de vista los \hyperref[sec:obj]{objetivos} y \hyperref[sec:usu]{usuarios diana} a utilizar este trabajo.  ha sido una tarea útil y fundamental contrastar las "soluciones" existentes para darme cuenta de que los datos no son nada accesibles ni para científicos para que puedan desarrollar estudios ni prácticamente para usuarios observadores ya que la antigüedad de los sistemas, el tiempo de carga y lo nula inversión en accesibilidad hace imposible su uso.

Tras ponerme en contacto con los servicios informáticos del instituto, no tuvieron problema alguno en pasarme todos los ficheros de los que se sirve el servidor, de hecho me llevé una grata sorpresa al ver que almacenaban más variables de las que ellos ofrecían, lo que me ha permitido hacer un trabajo más rico. En el capítulo 5: Análisis de los datos, lo analizaremos más en detalle.


Para terminar con mi propuesta, he recogido someramente los siguientes puntos en los que ha de guiarse la solución. Las principales deficiencias encontradas y objetivo a subsanar son:
\begin{itemize}
    \item Inexistencia de una interfaz de consulta para que usuarios externos puedan obtener los datos de forma uniforme y utilizando protocolos actuales.
    \item Los datos expuestos en la web no son completos, por razón que desconozco no se muestran todas las variables existentes.
    \item No permite realizar consultas conjuntas a los datos.
    \item La visualización gráfica no se consigue por todas las variables.
\end{itemize}


	% Análisis del problema
	% 1. Análisis de requisitos
	% 2. Análisis de las soluciones
	% 3. Solucion propuesta
	% 4. Análisis de seguridad
	\chapter{Análisis de los datos disponibles}

Los datos acerca de las defunciones son un componente fundamental en este trabajo sin el cual no podríamos construir nuestra solución. Si bien es cierto que no es difícil obtener estos datos del Instituto Nacional de Estadística, estos son demasiados pobres, no nos ofrecen información suficiente sobre para poder construir una solución con suficiente semántica.

Durante la documentación del capítulo anterior, descubrí que el Instituto Carlos III recolectaba e intentaba exponer estos datos con mucho más detalles que los anteriores. Tras conocer la existencia de los servidores Arïadna y Raziël me puse en contacto con el \textbf{Instituto de Salud Carlos III} por correo electrónico para que me compartieran estos datos de dominio público\footnote{Son los datos almacenados en el servidor interactivo Arïadna y Raziël comentados en el capítulo sobre el estado del arte. \hyperref[sec:estadoArte]{Enlace al capítulo}} almacenado en sus servidores. Esta colección de datos si es muy completa nos ofrece muchos tipos de enfermedades y nos ofrece infiormación normalizada.

Estas son las columnas normalizadas:
\begin{enumerate}
\item \textbf{AVP}: Años potenciales de vida perdidos.
\item \textbf{CRUDA}: Tasa bruta.
\item \textbf{TAVP}: Tasa de años potenciales de vida perdidos.
\item \textbf{EDAD}: Edad media a la defunción.
\item \textbf{TASAE}: Tasa ajustada a la población europea.
\item \textbf{TAVPE}: Tasa ajustada de años potenciales de vida perdidos.
\item \textbf{TASAW}: Tasa ajustada a la población mundial.
\item \textbf{TASAVPW}: Tasa ajustada de años potenciales de vida perdidos.
\end{enumerate}

El resto de columnas que forman los datos son:
\begin{enumerate}
\item \textbf{ANO}: Año de consulta (1980 a 2020).
\item \textbf{CAUSA}: Código de la causa.
\item \textbf{SX}: Sexo (1: hombres, 2: mujeres).
\item \textbf{CCAA}: Comunidad Autónoma.
\item \textbf{GEDAD}: Grupos de edad.
\item \textbf{DEFU}: Número de defunciones.
\end{enumerate}

Esta entidad es la que he denominado \textbf{Raziel} que integra otro tipo de modelos como la cauas (motivo de la defunción) que es un objeto formado por el nombre de la causa y su equivalencia con el estándar \gls{CIE-10} que es el acrónimo de la "Clasificación Internacional de enfermedades en su 10.ª edición."

La principal diferencia aparente que existe entre la información que sirve \textbf{Arïadna} y \textbf{Raziël} es que Arïadna no guarda información acerca de los grupos de edades, solo nos ofrece el cómputo de la edad media. Este cómputo podemos calcularlo nosotros conociendo los rangos de edad. Por tanto descarté Arïadna sospechando que es la misma fuente que Raziël pero con algunas columnas calculadas.
	
	\chapter{Planificación}
\label{sec:plani}
En este capítulo detallaré todo lo relativo a la planificación, cuantificación de recursos y
estimación del proyecto. 

\section{Metodología utilizada}
\label{sec:meto}
Desde que se empiezan a programar los primeros ordenadores en la historia, la programación
siempre ha sido minguneada y considerada bastante lejos de una actividad científica. Todo
empezó a cambiar cuando los scripts permitían realizar cálculos como ninguna otra
herramienta hasta el momento. Fue la matemática Margaret Hamilton quien planteó por
primera vez que los sistemas informáticos tenían que integrar tres componentes: hardware,
software y los usuarios que los iban a usar. Fue Margaret Hamilton, cuando se produce lo
que conocemos como crisis del software en 1968 quien acuñó en la \textit{NATO Software
Engineering Conference} el termino \textbf{Ingeniería} al proceso de creación de software. En ese
momento donde parecía que crear software duradero y escalable en el tiempo era misión
imposible hizo que se invirtiera en investigación y se construyera conocimiento. Este
conocimiento es lo que hoy llamamos Ingeniería de Software: herramientas, técnicas de
especificación y diseño que nos permiten especificar, desarrollar, validar y evolucionar
como en cualquier otra disciplina ingenieril.

El desarrollo ágil presenta una forma distinta de trabajar y organizarse, adaptándose
a las condiciones cambiantes que puedan surgir, aprovechando estos cambios para obtener
ventajas. Con este tipo de metodologías podemos dividir el trabajo en pequeñas piezas de
manera que podemos ir realizándolos de forma iterativa añadiendo valor concreto al
producto en cada iteración.

Se ha barajado utilizar una de las dos metodologías más empleadas en la industria: Scrum y
eXtreme Programming. eXtreme programming ha sido descartada por la imposibilidad de
aplicación en términos de tiempo y recursos humanos. Los roles y artefactos exigidos por
esta metodología son indiscutiblemente para un equipo de desarrollo segmentado. Sin
embargo y en comparación con Scrum, eXtreme programming es una metodología muy enfocada al proceso de
desarrollo. Además, obliga a desarrollar guiándose de pruebas, a programar en parejas y asegurar
la calidad del código en todas las etapas.

Scrum en cambio se muestra más flexible. En el año 2001 que K. Schwaber y Mike Beedle
publican el primer libro sobre Scrum \cite{agile_book}: Agile Software Development with
Scrum esta metodología se ha convertido en la más utilizada para el desarrollo de
software. Siendo precisos y prudentes tampoco es posible aplicar propiamente dicho Scrum
en este proyecto\ldots fundamentalmente por ser una persona a cargo de todo el proceso.

Por tanto, me he permitido crear mi propia metodología de desarrollo basándome en los tres
valores fundamentales que ofrecen estas tecnologías:
\begin{itemize}
    \item \textbf{Transparencia}: en todo momento se ha de conocer en qué se está
    trabajando, que problemas se está teniendo y/o si existe algún bloqueo asociado.
    \item \textbf{Inspección}: se ha de inspeccionar y no perder de vista el progreso para
    conseguir el objetivo. La trazabilidad del trabajo nos la ofrece el SCV \footnote{source
    control versioning} en nuestro caso git en GitHub con incorporación de funcionalidad por pull
    request.
    \item \textbf{Adaptación}: poder reaccionar a tiempo a los cambios requeridos por los
    stakeholders.
\end{itemize}


\section{Herramientas}
En esta sección voy a comentar las herramientas utilizadas para cumplir con los objetivos
metodológicos marcados en el anterior capítulo.

Para el contribuir con la segunda característica introspectiva citada anteriormente,
necesitamos no perder de vista el progreso para conseguir el objetivo. Se ha recurrido a
utilizar un sistema de control de versiones, sin discusión alguna GIT ha sido la
herramienta seleccionada. Se hace necesaria una forja donde respaldar nuestro código,
GitHub ha sido la opción seleccionada para alojar el código debido a lo ampliamente
utilizada que es la plataforma y las características premium que nos ofrece una cuenta universitaria como los
tableros Kanban, integración continua con GitHub Actions\ldots

Haciendo uso de esta herramienta, se han definido unas serie de milestones que agrupan
historias de usuario e \textit{issues} o tareas a realizar. Los incrementos en el trabajo se han
ido realizando por medio de \textit{Pull Request} y cada una de estas lleva enlazada una o varias
issues. Se puede ver el panel de issues en \ref{fig:issues-HU}.

\subsection{Milestones}
Los milestones hacen referencia al conjunto de productos mínimamente viables que se van
generando conforme se avanza en el desarrollo del proyecto. Un milestone está formado por
un conjunto de issues, se entiende finalizado un milestone cuando se han contemplado todos
los issues asociados. Los milestones creados se pueden observar en el repositorio de
GitHub y son los siguientes:
\begin{enumerate}
    \item \textbf{Infraestructura}: Puesta en marcha de la integración continua, creación
    de las primeras historias de usuario, tareas e investigaciones iniciales.
    \item \textbf{Modelo de datos}: definición de las clases que conforman el dominio del
    problema, atendiendo al estado del arte y primeras excepciones. 
    \item \textbf{Predicción y gráficos}: podemos comenzar a trabajar con el modelo de
    datos para realizar someras predicciones sobre los datos y obtener gráficas derivadas
    de la agrupación de éstos. He denominado \textit{managers} a la lógica software que
    realiza esta funcionalidad.
    \item \textbf{Acceso a recursos y a los managers}: supone la interfaz de comunicación
    agnóstica con la que podemos acceder a los datos y realizar las operaciones descritas
    en el milestone anterior de forma agnóstica utilizando el protocolo REST.
    \item \textbf{Sistema avanzado de consultas}: implementación del lenguaje de consulta
    y manipulación de datos para API GraphQL que permita consumir la API satisfaciendo las
    necesidades de los usuarios ficticios.
\end{enumerate}

La elaboración de la memoria y el pretexto de las decisiones técnicas elegidas 
se ha ido realizando de forma progresiva durante la realización de los milestones.


\subsection{Historias de usuario}
\label{sec:hu2}
Como estamos trabajando con una metodología de desarrollo ágil tenemos que expresar las
necesidades reales de los usuarios desde sus puntos de vista para lograr la
interacción del equipo de desarrollo con el usuario. Ello se consigue por medio de
historias de usuario, en adelante HU. Las HUs tienen que cumplir una serie de condiciones \cite{dddjj}:
\begin{itemize}
    \item Se tiene que identificar para que usuario se va a desarrollar la historia.
    \item La HU siempre está en el dominio del problema y siempre es narrada desde el
    punto de vista de ese usuario. Tiene que expresar el beneficio que obtendrá dicho
    usuario cuando se haya implementado la HU.
    \item La HU mayormente requiere la programación de cierta lógica de negocio. 
\end{itemize}
Todo lo que no cumpla estas características no son más que tareas enmarcadas en el proceso
de desarrollo. De las siguientes historias de usuarios se han generarán tareas específicas
cuya realización avanza en la HU.
\begin{itemize}
    \item \textbf{Como tribunal tengo que poder leer de forma ordenada y estructurada la
    memoria.} 
    Se enmarca en la naturaleza que tiene este trabajo de ser presentado y
    evaluado como TFG. Cuyo criterio de aceptación es: la elaboración de una memoria
    estructurada, detallada y ordenada que facilite el entendimiento del trabajo.

    \item \textbf{Trabajar con los datos de mortandad.}
    Como usuario programadora
    quiero poder acceder para trabajar a los datos de mortandad
    tal que pueda construir aplicaciones y servicios sin tener que preocuparme de montar cualquier infraestructura para obtener estos datos.

    \item \textbf{Predicciones y gráficos basados en los datos.}
    Como programadora quiero poder obtener predicciones sobre el incremento de defunciones
    tal que dada una enfermedad pueda obtener su predicción de defunciones en los próximos años.
\end{itemize}

Estas historias de usuarios que expresan una funcionalidad a alcanzar son ejecutadas
mediante tareas que en su completitud satisfacen el deseo del usuario. Se pueden observar en
el \href{https://github.com/pablojjimenez/TFG/issues}{tablero de issues} de GitHub.

\FloatBarrier
\begin{figure}[h]
	\centering
	\includegraphics[width=\textwidth]{doc/logos/imgs/issues.png}
	\caption{ Issues del TFG vistas desde GitHub. }
    \label{fig:issues-HU}
\end{figure}
\FloatBarrier


\subsection{Integración continua}
\label{sec:ci}
Una buena forma de seguir cumpliendo con las características \hyperref[sec:meto]{citadas
anteriormente}, a nivel de transparencia y adaptación es adecuado realizar el
desarrollo guiado por pruebas, lo que se conoce por TDD, \textit{Test Driven Development}.
Al emplear esta metodología garantizamos la calidad de lo programado, trasladamos los
requisitos a las pruebas de forma que se convierten en la más fiable documentación. Además
tener una gran cobertura de código testeado nos permite poder refactorizar con asiduidad y
garantizarnos no generar deuda técnica. 

La integración continua o CI \textit{continuos integration} es uno de los pilares de la
filosofía DevOps, se basa en realizar integraciones frecuentes con la máxima seguridad
posible ya que garantizan que el sistema se proteja siempre al incluir nuevos cambios ya
que estos deben de superar las \textit{pipelines} para terminar integrándose. 

Desde el inicio del trabajo se ha estado trabajando con un sistema férreo de CI de forma
que no se ha mezclado nunca nada que no haya superado los requisitos de diseño
especificados. La tecnología utilizada ha sido \textit{Github Actions} que es
una herramienta de CI que nos permite realizar una integración continua de la misma forma
que \textit{Travis CI} la elección de Github sobre Travis es por la rapidez del elegido e
integración con la forja utilizada que como se ha especificado anteriormente es Github.

Para garantizar la \textbf{calidad de la memoria} se han implementado dos tipos de CI\: un
revisor ortográfico y un revisor de estructura que compila el documento automáticamente y
pública el PDF generado en las
\href{https://github.com/pablojjimenez/TFG/releases}{\textit{releases} del repositorio.}

Del mismo modo, para el \textbf{código} se han implementado otras dos \textit{pipelines} que garantizan
que cuando se introduce en la rama principal del repositorio código nuevo, éste garantiza
unos estándares de calidad y que efectivamente el código es funcional por todas las
versiones especificadas de Python. Por un lado, para ausentar de errores el código fuente
y para comprobar la complejidad ciclomática. \footnote{Se ha utilizado la tecnología
\textbf{Flake8} por ser las más utilizada en el ecosistema. Algunos mensajes de error
pueden ser del tipo: \textit{library imported but unused} y \textit{Undefined name} } 

Por otro lado, se corre la \textit{suit} de tests de forma que se garantice que al añadir
una nueva parte de código no se rompe el existente. 

\FloatBarrier
\begin{figure}[h]
	\centering	
	\includegraphics[width=\textwidth]{doc/logos/imgs/CI-pr.png}
    \caption{Ejemplo de ejecución automática de la CI}
    \label{fig:tipos-de-cc}
\end{figure}
\FloatBarrier


\section{Planificación y cuantificación}
La planificación temporal realizada ha sido la siguiente.

\renewcommand{\arraystretch}{1.5}
\begin{table}[H]
	\centering
	\resizebox{\textwidth}{!}{%
	\begin{tabular}{@{}ccc@{}}
		\toprule
		\rowcolor[HTML]{ECF4FF} 
		\textbf{Productos mínimos viables} & \textbf{Fecha de comienzo} & \textbf{Fecha de finalización} \\ \midrule
	    \textbf{1. Infraestructura} & 14 de marzo & 28 de marzo \\
	    \textbf{2. Modelo de datos} & 29 de abril & 2 de mayo \\
	    \textbf{3. Predicciones y gráficos} & 4 de mayo & 15 de mayo \\
	    \textbf{4. Acceso a recursos y managers} & 17 de mayo & 5 de junio \\
	    \textbf{5. Sistema avanzado de consultas} & 6 de junio & 26 de junio \\
		\textbf{6. Despliegue y entrega} & 27 de junio & 8 de julio \\ \bottomrule
	\end{tabular}}
	\caption{Organización temporal del proyecto.}
\end{table}

Durante el trabajo se ha ido anotando las horas dedicadas a cada uno de los milestones. En
el gráfico se añade también la última etapa de despliegue de la aplicación y trabajo final
antes de la entrega.

\FloatBarrier
\begin{figure}[h]
	\centering	
	\includegraphics[width=\textwidth]{doc/logos/imgs/horas.png}
    \caption{Porcentaje de tiempo dedicado a las distintas etapas.}
    \label{fig:tipos-de-cc}
\end{figure}
\FloatBarrier

\subsection{Estimación de costes}
Es esta última sección vamos a detallar los costes asociados que tendría la elaboración
del proyecto.

\begin{table}[H]
\centering
    \begin{center}
        \begin{tabular}{| p{0.3\linewidth} | p{0.2\linewidth} | p{0.35\linewidth}|}
            \hline
            \rowcolor[HTML]{ECF4FF} 
            \textbf{Concepto} & \textbf{Coste} & \textbf{Comentarios} \\ \hline
            \textbf{MacBook Pro 2018} & 150 €/año & La cantidad de amortización aplicable es del 26\% a
        10 años. Con coste de compra: 1500€ \\
        \textbf{Recursos software} & 0 € &  \\
        \textbf{Recursos para el despliegue } & 23,94 €/mes & Puede consultarse en mayor
        detalle en \hyperref[sec:despliegue]{el capítulo 5, sección sobre el coste.}  \\
        \hline
        \end{tabular}
        \caption{Presupuesto del proyecto.}
    \end{center}
\end{table}




	% Desarrollo bajo sprints: 
	%   1. Permitir registros y login de usuarios
	%   2. Desarrollo del sistema de incidencias
	%   3. Desarrollo del sistema de denuncias administrativas y accidentes
	%   4. Desarrollo del sistema de croquis
	%   5. Instalación de la aplicación de manera automática
	\chapter{Desarrollo de la solución}
A la hora de empezar con la Implementación de la solución, se debe tener en cuenta los requisitos del sistema de acuerdo con los usuarios diana.

\subchapter{Diseño de la solución}
Para el diseño del software se ha utilizado el diseño dirigido por el dominio, DDD \textit{Domain Driven Design}. Este tipo de arquitectura introducida por Eric Evans \cite{ddd_book}: "Domain-Driven-Design - Tackling Complexity in the Hearth of Software, 2004" nos permite organizar el código de forma separada y organizada lo que nos permite poder realizar TDD correctamente por tener las distintas capas separadas y débilmente acopladas unas de las otras. La lógica de negocio reside en un único sitio y lo más cerca del dominio posible lo que nos permite cumplir los principios SOLID fácilmente y conseguimos un código abierto a su extensión pero cerrado a su modificación. 

\section{DDD}
En este tipo de arquitectura podemos diferenciar 3 elementos.

\subsection{Entidad}
Las entidades son aquellas clases que representan una entidad del dominio. Estas clases deben tener una estructura de datos que represente la información de la entidad y unas responsabilidades concretas. Se caracterizan porque tienen que ser consideradas iguales a otros objetos aún cuando cuando sus atributos difieren.

\subsection{Value objects}
Representan los coneptos que no tienen una entidad. Describen características por lo que solo nos interesan sus atributos.
Los value object representan elementos del modelo que se describen por el \textbf{qué} son, y no por \textbf{quién} o \textbf{cuál} son.

\subsection{Servicios}
No tienen significado propio suponen la capa que representa las operaciones que no pertenecen conceptualmente a ningún objeto del dominio concreto.


\subchapter{Implementación}
\section{Lenguaje de programación}
El lenguaje elegido ha sido \textbf{Python} los motivos de esta decisión son variados, por los que procedo a enumerarlos:
\begin{itemize}
    \item Es un lenguaje que se encuentra en las \cite{tiobe} primeras posiciones como lenguaje más utilizado a día de hoy. Además me encuentro muy cómodo programando en él.
    \item Es un lenguaje que cuenta con muchísimas bibliotecas, muchas de estas especiales para trabajar con datos y ficheros (como \textbf{Pandas}). Abundan también los \textit{frameworks} que nos ayudan a construir APIs con él y la mayoría de PaaS \footnote{Platform as a Service} soportan este lenguaje.
    \item Es un lenguaje interpretado, débilmente tipado y con una capacidad introspectiva muy alta que nos permite metaprogramar.
\end{itemize}

A la hora de realizar un proyecto software siempre es necesario utilizar bibiotecas y código implementado por otras personas para poder llegar más lejos. Todo esto es lo que llamamos dependencias. Muchos entornos de programación tienen gestores de dependencias que nos permiten instalar y gestionar estas dependencias cómodamente. En el entorno JavaScript tenemos \textit{npm} o \textit{yarn} que nos permite instalar dependencias de forma sencilla. Para la JVM el tradicional \textit{maven} es una buena opción o la mas moderna y potente \textit{Gradle}.

En Python se ha extendido el uso de \textbf{Poetry} que de forma muy parecida a \textit{npm} genera un fichero de configuración \textit{pyproject.toml} donde definimos las dependencias, las versiones y los paquetes de nuestro proyecto así como algunas tareas específicas para correr los tests o el lint fácilmente. Esto facilita enormemente la creación de entornos virtuales de prueba y de desarrollo tanto el local como por el CI.

\section{Continuos Integration}
Desde el inicio del trabajo se ha estado trabajando con un sistema ferreo de CI de forma que no se ha mezclado nunca nada que no haya superado los requisitos de diseño especificados. En el entorno de desarrollo se ha utilizado \textit{Github Actions} que es una herramienta de CI que nos permite realizar una integración continua de la misma forma que \textit{Travic CI} la elección de Github sobre Travis es por la rápidez del elegido e integración con la forja utilizada que como se ha especificado anteriormente es Github.

Para garantizar la \textbf{calidad de la memoria} se han implementado dos tipos de CI\footnote{Continuos Integration}: un revisor ortográfico y un revisor de estructura que compila el documento automáticamente y pública el PDF generado en las \href{https://github.com/pablojjimenez/TFG/releases}{\textit{releases} del repositorio.}

Del mismo modo, para el \textbf{código} se han implementado otras dos CI que garantizan que cuando se introduce en la rama principal del repositorio código nuevo, éste garantiza unos estándares de calidad y que efectivamente el código es funcional por todas las versiones especificadas de Python. Por un lado, verificar el código fuente contra errores de programación (como "library imported but unused" y "Undefined name") y para verificar la complejidad ciclomática. Se ha utilizado la tecnología \cite{flake}\textbf{Flake8} por ser las más utilizada en el ecosistema.
Por otro lado, se corre la \textit{suit} de tests de forma que se garantice que al añadir uno código no se rompe el existente. Se ha utilizado la librería \textit{Unittest} que se encuentra incluída en la libreria estandar de Python permite realizar aserciones, provee de muchas aserciones y permite organizar los tests de forma parecida a como lo hace jUnit en clases que llamamos \textit{suits} de tests.

Todas estas comprobaciones se realizan para todas las versiones de Python en las que garantizamos que el software funciona.

	\chapter{Despliegue de la aplicación}
\label{sec:cap6}
En la etapa de despliegue es cuando el software queda disponible para el uso y disfrute de
los usuarios. Supone una etapa fundamental para dar por finalizada la construcción del
software. 

A la hora de elegir un modelo de entorno cloud tenemos distintos modelos de cloud
computing, dependiendo de las partes de la infraestructura local que queramos gestionar.

\FloatBarrier
\begin{figure}[h]
	\centering	
	\includegraphics[width=\textwidth]{doc/logos/imgs/iaas-paas.png}
    \caption{ Distintos modelos de computación en la nube.
	\href{https://www.redhat.com/es/topics/cloud-computing/iaas-vs-paas-vs-saas}{RedHat -
	CC}}
    \label{fig:tipos-de-cc}
\end{figure}
\FloatBarrier


Necesito que el modelo de computación en la nube me provea de facilidad suficiente para
añadir y desplegar nueva funcionalidad con rapidez. El coste tiene que ser lo más
inteligente y adaptado posible. Y además, quiero un sistema que me prevea de una
plataforma ya estructurada de forma que solo tenga que centrarme en la configuración
correcta de mi servidor de acorde con los requisitos de la plataforma sin tener que
configurar ninguna infraestructura de bajo nivel como los puertos del sistema, el sistema
operativo\ldots


Estos requisitos hacen que utilizar como entorno cloud un PaaS, \textit{Platform as a
service} sea la mejor opción. Este me proveerá de varias capas de servicios apilados que me va a permitir ejecutar y
gestionar mi aplicación desplegada sin tener que mantener la infraestructura subyacente.


En el mercado existen distintas empresas que nos ofrecen este tipo de servicios, podemos
destacar: \href{https://aws.amazon.com/es/elasticbeanstalk/}{AWS Elastic Beanstalk},
\href{https://dashboard.heroku.com/login}{Heroku} o \href{https://vercel.com/}{Vercel}
entre otros. Mientras que Vercel no soporta (por el momento) despliegues basados en imágenes de docker,
que es la \hyperref[sec:proceso-despliegue]{forma de despliegue que persigo}, y AWS
solicita información bancaria, he utilizado Heroku.

\section{Proceso de despliegue}
\label{sec:proceso-despliegue}
Para poder desplegar una aplicación en Heroku, es necesario acceder al panel de
administración y crearnos una app. Sin embargo, existe un CLI \footnote{Interfaz de
línea de comandos} que se instala en nuestro sistema y nos permite realizar de forma más
cómoda las configuraciones necesarias.

A la hora de desplegar el código, Heroku nos ofrece distintas formas, una de ellas es
poner a la escucha una rama del repositorio de GitHub a cualquier tipo de cambio para
desplegarlo. Sin embargo, este proyecto tiene muchos \textit{assets}, los ficheros con
los datos de las defunciones que no están versionados en GitHub. Este
motivo y el poco margen de personalización que provoca este método sobre la
configuración de nuestra aplicación me hizo desestimar esta forma.

El método que he seguido ha sido crear un contenedor de la aplicación que luego publicamos
en la app creada en Heroku. Los contenedores, entre otras cosas, nos ofrecen tener una portabilidad absoluta
ya que podemos poner en marcha la aplicación en cualquier sistema operativo sin
preocuparnos de las dependencias necesarias y eso es lo que necesitamos hacer en el
servidor de Heroku.

El contenedor y su configuración se define como código en un archivo de texto plano
llamado \codeword{Dockerfile} en él se define el software necesario a instalar, las
dependencias del proyecto, se expone el host, el puerto y el número de \textit{workers}
\footnote{Son el número de procesos que va a ejecutar el dyno. En la versión gratuita de Heroku solo aprovechamos 2 \textit{workers} por
limitación del hardware.} a
ejecutar como variables de entorno que me permitirán a través del PaaS configurar o cambiar.
Por último, se especifica en el comando que ejecuta el contenedor.

\FloatBarrier
\begin{figure}[h]
	\centering	
	\includegraphics[width=\textwidth]{doc/logos/imgs/deployd.png}
	\caption{ Arquitectura del despliegue. }
    \label{fig:arq-deploy}
\end{figure}
\FloatBarrier

\begin{itemize}
    \item \textit{Dyno} es un contenedor liviano con Linux en los que Heroku ejecuta la
    aplicación.
\end{itemize}

Utilizando la CLI podemos ver el sistema de registros (que también está disponible en el
\textit{dashboard} web) con el que podemos monitorizar todas las peticiones que se
realizan así como los reinicios de la máquina virtual y el estado del servidor.

\FloatBarrier
\begin{figure}[h]
	\centering	
	\includegraphics[width=\textwidth]{doc/logos/imgs/logs.jpg}
	\caption{ \textit{Logs} de la aplicación desplegada. }
    \label{fig:heroku-logs}
\end{figure}
\FloatBarrier

\section{Coste del despliegue}
\label{sec:despliegue}

Durante el proceso de desarrollo y las primeras pruebas se ha utilizado la modalidad ``Free
\& Hobby'' que nos permite desplegar la aplicación de forma gratuita con la limitación de
que la máquina entra en estado de reposo tras 30 minutos de inactividad, lo que
hace que cuando se reciba cualquier petición se encienda la máquina virtual que contiene
el contenedor con la aplicación, además los dynos \footnote{Es un contenedor liviano con Linux en los que Heroku ejecuta la
aplicación.}
tienen 2 cores por lo que nos limita el número de procesos.

Para la puesta en producción se debería de contratar la tarifa ``Standard 1X'' que nos ofrece
512MB de RAM, con ilimitado número de procesos gracias al autoescalada y nos incluye un
sistema de métricas y avisos que no está disponible en la versión gratuita. Esta tarifa
tiene un coste de 25\$ al mes.


	\input{doc/secciones/08_conclusiones.tex}
	
	\newpage
	\bibliography{bibliografia}
	\bibliographystyle{plain}
	
\end{document}

